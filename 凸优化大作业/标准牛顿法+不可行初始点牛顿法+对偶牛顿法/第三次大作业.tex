\documentclass{article}
\usepackage[utf8]{inputenc}
\usepackage{ctex}
\usepackage{amsmath}
\usepackage{indentfirst}
\usepackage{amssymb}%花体字符
\usepackage{graphicx}
\usepackage{geometry}
\usepackage{enumerate}
\usepackage{float} 
\usepackage{subfigure}
\usepackage{fontspec}
\usepackage{listings}
\usepackage[framed,numbered,autolinebreaks,useliterate]{mcode}

\geometry{a4paper,scale=0.8}
\title{第三次大作业}
\author{计试001 苏悦馨 2204120515}
\date{}

\begin{document}
\maketitle
\section{问题简述}
等式约束熵极大化问题
\begin{equation*}
    \begin{array}{ll}
        \min & f(x)=\sum\limits_{i=1}^n x_i logx_i \\
        s.t. & Ax=b
    \end{array}
\end{equation*}
其中$dom\ f=R^n_{++},A \in R^{p \times n},p<n ,rank(A)=p$,问题实例为$n=100,p=30$

随机生成一个满秩矩阵A,随机选择一个正向量作为$x_0$,然后令$b=Ax_0$
采用以下方法计算该问题的解
\begin{enumerate}[a)]
    \item 标准$Newton$方法,初始点为$x_0$
    \item 不可行初始点$Newton$方法,初始点为$x_0$,以及随机生成的向量$v_0$
    \item 对偶$Newton$方法
\end{enumerate}

\section{问题分析}
首先随机生成满秩矩阵$A$和正向量$x_0$
\begin{lstlisting}
    %随机生成满秩矩阵A
    A=rand(p,n);
    while rank(A) ~= p
        A=0.1+rand(p,n);
    end
    %随机生成正向量 x0
    x0=rand(n,1);
    b=A*x0;  %x0是可行的
\end{lstlisting}

使用$matlab$符号函数计算$\nabla^2 f, \nabla f, r(x,v)$
\begin{lstlisting}
    %生成原函数以及其他计算中用到的函数
    syms X [n,1] matrix 
    syms V [p,1] matrix
    f=X.'* log(X);%原函数f
    H_f=diff(f,X,X.');%函数f的hessian矩阵
    G_f=diff(f,X)';%函数f的梯度
    r=[G_f+A'*V;A*X-sym(b)];%原对偶残差
\end{lstlisting}

\subsection{标准$Newton$方法}
由可行初始点的标准$Newton$方法,下降方向$d^k_x$满足
\begin{equation*}
    \left[\begin{array}{cc}
            \nabla^2 f(x) & A^T \\
            A             & 0
        \end{array}\right]
    \left[\begin{array}{c}
            d^k_x \\
            w
        \end{array}\right]
    =
    \left[\begin{array}{c}
            -\nabla f(x) \\
            0
        \end{array}\right]
\end{equation*}

牛顿减少量$\lambda^2(x^k)$满足
\begin{equation*}
    \lambda^2(x^k)={d^k_{nt}}^T \nabla^2 f(x) d^k_{nt}
\end{equation*}

实现代码如下
\begin{lstlisting}
    value_G_f=double(sym(subs(G_f,x)));
    value_H_f=double(sym(subs(H_f,x)));
    matrix_A=[value_H_f,A';A,zeros(p,p)];
    matrix_b=[-value_G_f;zeros(p,1)];
    dr=matrix_A\matrix_b;
    d=dr(1:100);%新的牛顿方向
    w=dr(101:130);%新的对偶变量
\end{lstlisting}

回溯直线搜索部分如下
\begin{itemize}
    \item 外循环:判断$\frac{1}{2}\lambda^2(x^k)<10^{-10}$
    \item 内循环:通过回溯直线搜索求解$t^k$,使得$f(x^k+t^kd^k)<f(x^k)-\alpha t^k \lambda^2(x^k)$
          \begin{lstlisting}
    while double(sym(subs(f,x_))) > (double(sym(subs(f,x)))-alpha*t*lamda_2)
        t=beta*t;
        x_=x+t*d;
    end
    \end{lstlisting}
    \item 在每次内循环结束后,更新$x^{k+1},\nabla f(x^{k+1}) ,\nabla^2 f(x^{k+1}),\lambda^2(x^{k+1})$
          \begin{lstlisting}
    %更新参量
    x=x+t*d;%新的x

    value_G_f=double(sym(subs(G_f,x)));
    value_H_f=double(sym(subs(H_f,x)));
    matrix_A=[value_H_f,A';A,zeros(p,p)];
    matrix_b=[-value_G_f;zeros(p,1)];
    dr=matrix_A\matrix_b;
    d=dr(1:100);%新的牛顿方向
    w=dr(101:130);%新的对偶变量
    lamda_2=d'*value_H_f*d;%新的牛顿减少量
    \end{lstlisting}
\end{itemize}

\subsection{不可行初始点$Newton$方法}
由不可行初始点的标准$Newton$方法,下降方向$d^k_x$满足
\begin{equation*}
    \left[\begin{array}{cc}
            \nabla^2 f(x) & A^T \\
            A             & 0
        \end{array}\right]
    \left[\begin{array}{c}
            d^k_x \\
            d^k_v
        \end{array}\right]
    =
    -\left[\begin{array}{c}
            \nabla f(x)+A^Tv \\
            Ax^k-b
        \end{array}\right]
\end{equation*}

实现代码如下
\begin{lstlisting}
    value_G_f=double(sym(subs(G_f,x)));
    value_H_f=double(sym(subs(H_f,x)));
    matrix_A=[value_H_f,A';A,zeros(p,p)];
    matrix_b_2=-[value_G_f+A'*v;A*x-b];
    dr_2=matrix_A\matrix_b_2;
    dx=dr_2(1:100);%牛顿方向
    dv=dr_2(101:130);%对偶变量更改量
\end{lstlisting}

对原对偶残差$\Vert r \Vert_2$回溯部分如下
\begin{itemize}
    \item 外循环:判断$\frac{1}{2}\lambda^2(x^k)<10^{-10}$
    \item 内循环:通过回溯直线搜索求解$t^k$,使得$f(x^k+t^kd^k)<f(x^k)-\alpha t^k \lambda^2(x^k)$
          \begin{lstlisting}
    while norm(double(sym(subs(r,{X,V},{x_,v_})))) > (1-alpha*t)*norm_r
        t=beta*t;
        x_=x+t*dx;
        v_=v+t*dv;
    end
    \end{lstlisting}
    \item 在每次内循环结束后,更新$x^{k+1},\nabla f(x^{k+1}) ,\nabla^2 f(x^{k+1}),\Vert r(x^{k+1},v^{k+1}) \Vert_2$
          \begin{lstlisting}
    %更新参量
    x=x+t*dx;%新的x
    v=v+t*dv;% 新的v

    value_G_f=double(sym(subs(G_f,x)));
    value_H_f=double(sym(subs(H_f,x)));
    matrix_A=[value_H_f,A';A,zeros(p,p)];
    matrix_b_2=-[value_G_f+A'*v;A*x-b];
    value_r=double(sym(subs(r,{X,V},{x,v})));
    norm_r=norm(value_r);%初始点处原对偶残差的范数
    dr_2=matrix_A\matrix_b_2;
    dx=dr_2(1:100);%牛顿方向
    dv=dr_2(101:130);%对偶变量更改量
    \end{lstlisting}
\end{itemize}

\subsection{对偶$Newton$方法}
首先求出该问题的对偶问题,记$A^{p \times n}=[a_1,a_2,\dotsb,a_n]$

该问题的$lagrange$函数为
\begin{equation*}
    L(x,v)=\sum\limits_{i=1}^n x_i logx_i + v^T(Ax-b)
\end{equation*}

由于$lagrange$函数为凹函数,则$\frac{\partial L(x,v)}{\partial x_i}=0,\ i=1,2,\dotsb,n$时,$g(v)=\inf\limits_x L(x,v)$

由
\begin{equation*}
    \frac{\partial L(x,v)}{\partial x_i} = \log x_i +1+v^Ta_i
\end{equation*}
解得
\begin{equation*}
    x_i=e^{-(1+v^Ta_i)},\ i=1,2,\dotsb,n
\end{equation*}

则代入$x_i$,得到对偶函数为
\begin{align*}
    g(v)=-e^{-1}\sum\limits_{i=1}^n e^{-v^Ta_i}-b^Tv
\end{align*}

则得到原问题的对偶问题为
\begin{equation*}
    \min\limits_{v\in R^p}  \  e^{-1}\sum\limits_{i=1}^n e^{-v^Ta_i}+b^Tv
\end{equation*}
\\ \hspace*{\fill}

对于该无约束优化问题,使用$Newton$下降方法如下(记$f(v)=-g(v)$)

由牛顿方法,下降方向$d^k_{nt}$满足:
\begin{align*}
    d^k_{nt} & =-(\nabla^2 f(v^k))^{-1} \nabla f(v^k) \\
\end{align*}

牛顿减少量$\lambda^2(x^k)$满足
\begin{equation*}
    \lambda^2(x^k)={d^k_{nt}}^T \nabla^2 f(x) d^k_{nt}
\end{equation*}
其中
\begin{align*}
    \nabla f(v)=b-Ae^{-1-A^T v}\\
    \nabla^2 f(v)=A\ diag(e^{-A^Tv-1})\ A
\end{align*}
\\ \hspace*{\fill}

代码思路:
\begin{enumerate}[a)]
    \item 使用matlab符号函数(syms)定义$-g(v)$,加负号是为了变成凸优化问题。
    \begin{lstlisting}
    g=sym(ones(1,n))*exp(-sym(ones(n,1))-sym(A')*V)+sym(b')*V; 
    \end{lstlisting}
    \item 外循环:判断$\frac{1}{2}\lambda^2(v^k)<10^{-10}$
    \item 内循环:通过回溯直线搜索求解$t^k$,使得$f(v^k+t^kd^k)<f(v^k)-\alpha t^k \lambda^2(v^k)$
    \begin{lstlisting}
    while double(sym(subs(g,v_))) > (double(sym(subs(g,v)))-alpha*t*lamda_v)
        t=beta*t;
        v_=v+t*d_v;
    end
    \end{lstlisting}
    \item 在每次内循环结束后,记录$f(v^k)-p^\ast$,更新$v^{k+1},\lambda^2(v^k)$
    \begin{lstlisting}
    v=v+t*d_v;
    g_v=b-A*exp(-1-A'*v);
    hessian_v=A*diag(exp(-1-A'*v))*A';
    d_v = -hessian_v\g_v;%下降方向
    lamda_v= d_v'*hessian_v* d_v; %牛顿减少量的平方
    k=k+1;
    \end{lstlisting}
\end{enumerate}

\section{结果分析}
\subsection{证实求得相同的最优点}
运行程序后分别保存求得的最优变量$x,v$的值(迭代中不涉及求某一变量求值的,代入$x,v$的相关表达式计算),后通过计算$\Vert x_i -x_j \Vert_2,\ i,j=a,b,c,\ \ \Vert v_i -v_j \Vert_2,\ i,j=a,b,c$来验证是否求得了相同的最优点。

代码如下
\begin{lstlisting}
    norm(x_a-x_b)
    norm(x_b-x_c)
    norm(x_c-x_a)

    norm(v_a-v_b)
    norm(v_b-v_c)
    norm(v_c-v_a)
\end{lstlisting}
最优解之间的$Euclid$距离为(对应上述程序)\\
$[9.0099e-06,
1.4022e-06,
9.0982e-06,
2.4200e-11,
9.9998e-07,
9.9998e-07]$\\
由于计算误差的存在,最优解之间极小的的$Euclid$距离是被允许的,从而可证实求得了相同的最优点。

\subsection{迭代次数比较}
\begin{table}[H]
    \begin{center}
    \begin{tabular}{l|c|r} % <-- Alignments: 1st column left, 2nd middle and 3rd right, with vertical lines in between
    方法 & 迭代次数 \\
    \hline
    标准$Newton$方法 & 6 \\
    不可行初始点$Newton$方法 & 8 \\
    对偶$Newton$方法 & 9 \\
    \end{tabular}
    \end{center}
\end{table}

可以看出,使用不同的方法求解该问题,在从同一初始点出发,到达指定的精度的情况下,迭代次数并没有太大的差别。

\end{document}